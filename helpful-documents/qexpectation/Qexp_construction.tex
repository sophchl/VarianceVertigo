\documentclass{article}
\usepackage[utf8]{inputenc}
\usepackage{geometry}
    \geometry{
    a4paper,
    total={170mm,257mm},
    left=20mm,
    top=20mm,
    }
\setlength\parindent{0pt}
\usepackage{amsmath}
\usepackage{amssymb}
\newcommand*\diff{\mathop{}\!\mathrm{d}}

\title{How to compute the option implied volatility}
%\author{}
%\date{April 2020}

\begin{document}

\maketitle

\section{Context}\label{sec:1}

There are 2 volatilities to be computed: the expectation under P and the expectation under Q.\\

\textbf{Expectation under P}\\
This is the expectation calculated using historic return data.\\

The authors use a random walk model 
\begin{align}\label{eq:Ex_p}
\mathbb{E}_{t}^{\mathbb{P}} \left[ RV_{t,h}^{U/D} (\kappa) \right] = RV_{t-h,h}^{U/D}(\kappa)
\end{align}

where the realized variance is calculated per day, and summed up over a time horizon h
\begin{align}\label{eq:RV_p}
RV_{t}^{U/D}(\kappa) = \sum_{j=1}^{n_{t}} r_{j,t}^{2} [\text{indicator above/below $\kappa$}]\\
RV_{t,h}^{U/D} = \sum_{j=1}^{h} RV_{t+j}^{U/D}(\kappa)
\end{align}

Also the authors briefly mention, that this formula converges to the instantaneous variance, assuming that the stock price follows a jump process. However, we only have to compute the sums above. So I don't think we have to deal much with jumps.\\

\textbf{Expectation under Q}\\
This is the expectation calculated using option data. The authors use so called model-free/corrior implied volatility, hence they do not make assumptions regarding the dynamics of the underlying.

\begin{align}\label{eq:Ex_q}
IV_{t,h}^{U} = \mathbb{E}_{t}^{\mathbb{Q}} \left[ RV_{t,h}^{U} (\kappa) \right] \approx 2 \int_{F_{t}epx(\kappa_{F}}^{\infty} \frac{M_{0}(K)}{K^{2}} \diff K,\\

IV_{t,h}^{D} = \mathbb{E}_{t}^{\mathbb{Q}} \left[ RV_{t,h}^{D} (\kappa) \right] \approx 2 \int_{0}^{F_{t}epx(\kappa_{F}} \frac{M_{0}(K)}{K^{2}} \diff K
\end{align}
with
\begin{align*}
M_{0}(K) = min(P_{t}(K), C_{t}(K))
\end{align*}
and with $\kappa_{F} = (\kappa - r_{t}^{f})$ to get consistent thresholds when computing realized and option-implied volatility .\\

For the intuition and derivation of this formula: see other QExp document.


\section{Information from online appendix}\label{sec:2}

Description of procedure
\begin{enumerate}
\item preprocessing
\begin{enumerate}
	\item sort call and put options by maturity and strike price
	\item average bid-ask quotes for each contract
	\item preprocess: apply filters as in Chang, Christoffersen, Jacobs (2013) (see document Stefano)
	\item only consider OTM contracts
	\item discard call with moneyness $<$ 0.97, put with moneyness $>$ 1.03
\end{enumerate}
\item construct continuous grid 
\begin{enumerate}
	\item for each day and each maturity: interpolate implied volatilities over moneyness domain, using cubic spline (use only days with at least 2 OTM calls and 2 OTM puts)
	\item outside of observed moneyness domain: extrapolate
	\item obtain: 1,000 IV between moneyness of $0.01$\% and $300$\%.
\end{enumerate}
\item construct moments
\begin{enumerate}
	\item map IV to call and put prices: call for moneyness $>$ 1, put for moneyness $<$ 1
	\item approxmiate integrals: Lobatto quadrature
	\item linear interpolation for corresponding moments
\end{enumerate}
\end{enumerate}

Other things to consider
\begin{itemize}
\item need comparable numbers of OTM puts and calls in longer-horizon maturities
\end{itemize}

How to construct IV?
\begin{itemize}
\item threshold: $\kappa = 0$, process to risk-free: $b = F_{t}exp(\kappa)$
\item model-free corridor risk-neutral volatility as in Andersen, Bondarenko, Gonzalez-Perez (2015), Andersen and Bondarenko (2007), Carr, Madan (1999)
\end{itemize}


\section{What are the acutal steps and computations?}\label{sec:3}

\textbf{P-Expectation}\\

$\mathbb{E}_{t}^{\mathbb{P}} \left[ RV_{t,h}^{U/D} (\kappa) \right] $: compute the sum as in equation \ref{eq:RV_p}.\\

see document: Realized Variance for cleaning and other steps.\\

\textbf{Q-Expectation}\\

$\mathbb{E}_{t}^{\mathbb{Q}} \left[ RV_{t,h}^{U/D} (\kappa) \right]$. try to find a way to compute the integral from equation \ref{eq:Ex_q}.\\

\begin{enumerate}
\item preprocess the data (step 1 above/Stefanos document)
\item compute the integral above
\item interpolate and extrapolate (step 2 above)

\end{enumerate}

\section{Remaining Questions}
(all references of sections and step refer to this document)
\begin{enumerate}
\item section \ref{sec:2}, step 2: So this they do after they constructed the IV? Sounds to me as if they skip the most important step.
\item section \ref{sec:2}, step 3: Why to they map IV to put and call prices, what integrals do they compute? If they already have the IV, there are no more integrals to compute, or?
\item section \ref{sec:3}, step 2: How will we compute the integrals? They mention Lobatto quadrature, Nikola refered us to the computation of the VIX.
\end{enumerate}

\end{document}