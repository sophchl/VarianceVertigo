\documentclass{article}
\usepackage[utf8]{inputenc}

\title{Excess returns}
\author{3S team}
%\date{gyjugyu}

\begin{document}
\maketitle


Path for the equity return: CRSP$\rightarrow$Index / S\&P 500 Indexes$\rightarrow$Index File on the S\&P 500  $\rightarrow$ Return on S\&P Composite Index.\\
Path for the risk-free rate: CRSP $\rightarrow$ Treasuries $\rightarrow$ Riskfree Series (1-month and 3-month). Then we chose 3-months.\\
For both of them we selected monthly returns and we observed that the risk-free is already annualized. For the excess return, we computed:
$$
    \sum_t(12\times\ln(1+R_t)) - \sum_t \ln(1+R_t^f),
$$
where $t$ are the months in the dataset, 12 is the annualization factor for log returns for the equity returns, $R_t:=\frac{S_t}{S_{t-1}}$ is the equity return and $R_t^f$ is the risk-free rate.\\
The reason why we decided to compute returns in this way it's simply because, as far as we know, to aggregate log excess return we have to aggregate the 'total-log-return' and then to subtract the 'total-risk-free-log-return'

\end{document}
