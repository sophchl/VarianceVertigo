\documentclass{article}
\usepackage[utf8]{inputenc}
\usepackage{geometry}
    \geometry{
    a4paper,
    total={170mm,257mm},
    left=20mm,
    top=20mm,
    }
\setlength\parindent{0pt}
\usepackage{amsmath}
\newcommand*\diff{\mathop{}\!\mathrm{d}}

\title{Volatility indices under the risk-neutral measure}
%\author{}
%\date{April 2020}

\begin{document}

\maketitle

\section{Chang et al (2013) Appenndix A: Extracting option-implied moments}

\section{Andersen, Bondarenko (2007): Construction and Interpretation of the model-free implied volatility}

\subsection{Introduction}

different volatility indices
\begin{itemize}
    \item historic volatility (HV): use past realized volatility to estimate future realized volatility
    \item Black-Scholes implied volatility (BSIV): use the BS model, enter option prices, solve for volatility
    \item Corridor implied volatility (CIV): extract volatility from option prices but truncate the strike prices used at a barrier
    \item Model-free implied volatility (MFIV): extract volatility from option prices using only no-arbitrage arguments and a full range of strike prices 
\end{itemize}

\textit{Note: I remember having read elsewhere (sorry unscientific quoting) that even though the indexes are called ``volatility'' indexes, they rather measure the return variation/quadratic variation than the volatility of the underlying stochastic process, which is unobserved. I think however they are a good estimate for the volatility.}\\

Discussion MFIV
\begin{itemize}
\item ``aims to measure the expected integrated variance, or more generally return variation, over the coming month, evaluated under the so-called risk-neutral or pricing (Q) measure''
\item no assumptions made regarding underlying return dynamics in contrast to BSIV
\item however, needs strikes spanning full range of possible values for underlying asset - data requirements often not met, hence approximations have to be made
\item MFIV will often differ from return variation under P, hence not a pure volatility forecast but adds uncertainty surrounding that forecast $\rightarrow$ will contain a premium compensating for exposure to equity-index volatility
\end{itemize}

\textit{Note: So MFIV contains a premium because it not only measure the uncertainty of the returns, but also the uncertainty regarding the return volatility? (see also Carr, Wu, 2008, summary in theoryquestion channel)}\\

Some dimensions that all volatility indices have to meet
\begin{itemize}
    \item risk-neutral density must satisfy NoA constraints, e.g. risk-neutral density must be positive
    \item how to handle tails of the distribution?
    \begin{itemize}
        \item CBOE's VIX: truncate the tail (according to the authors, that makes VIX more an imperfect CIV index than a MFIV measure)
        \item alternative: extend the risk-neutral density in the tails
    \end{itemize}
\end{itemize}

\subsection{Theoretical Background: Barrier variance and corridor variance contracts}

\textbf{Notation}

\begin{itemize}
	\item $P_{t}(K), C_{t}(K)$ .. call and put with strike K 
    \item $F_{t}$ .. S\&P 500 future contract at time t: they use $r_{f} = 0$ but use forward instead of spot prices, using the relationship $F_{t} = e^{r_{f}(T-t)} S_{t}$ 
    \item $k = K/F_{t}$ .. moneyness
\end{itemize}

\subsubsection{Derive relationship for NoA pricing of contracts}

Option-prices can be computed using risk-neutral density (RND) $h_{t}(F_{t})$
\begin{align*}
P_{t}(K) = E_{t}^{Q}[(K - F_{T})^{+}] = \int_{0}^{\infty} (K-F_{T})^{+} h_{t}(F_{T}) \diff F_{T} \\
C_{t}(K) = E_{t}^{Q}[(F_{T} - K)^{+}] = \int_{0}^{\infty} (F_{T} - K)^{+} h_{t}(F_{T}) \diff F_{T}
\end{align*}

RND is as in Breeden-Litzenberger (1978)
\begin{align*}
    h_{t}(F_{T}) = \frac{\partial^{2} P_{t}(K)}{\partial K^{2}} \big\rvert_{K = F_{T}} = \frac{\partial^{2} C_{t}(K)}{\partial K^{2}} \big\rvert_{K = F_{T}} 
\end{align*}

According to Carr-Madan (1998) we can write the paypoff $g(F_{T})$ (assuming finite second derivatives a.e.) as
\begin{align*}
g(F_{T}) = g(x) + g'(x)(F_{T} - x) + \int_{0}^{x} g''(K)(K - F_{T})^{+} \diff K + \int_{x}^{\infty} g''(K)(F_{T} - K)^{+} \diff K
\end{align*}

set $x = F_{0}$ and take the expectation and setting $x = F_{0}$
\begin{align}\label{eq:po1}
	E_{0}^{Q}[g(F_{T})] = g(F_{0}) + \int_{0}^{F_{0}} g''(K)P_{0}(K) \diff K + \int_{F_{0}}^{\infty} C_{0}(K)  \diff K\\
	\Leftrightarrow
    E_{0}^{Q}[g(F_{T})] = g(F_{0}) + \int_{0}^{\infty} g''(K)M_{0}(K) \diff K
\end{align}
with $M_{0}(K) = min(P_{t}(K), C_{t}(K))$ (hence the OTM option).\\

\textit{Note: No expectation on the RHS because RHS is all $F_{0}$ measurable?}\\

In the current setting, $F_{t}$ is a martingale under $Q$. Assume the diffusion
\begin{align}
    \frac{\diff F_ {t}}{F_{t}} = \sigma_{t} \diff W_{t}
\end{align}

By Itos Lemma we get for the payoff
\begin{align*}
    g(F_{T}) = g(F_{0}) + \int_{0}^{T} g'(F_{t}) \diff F_{t} + \frac{1}{2}\int_{0}^{T} g''(F_{t})F_{t}^{2} \sigma_{t}^{2} \diff t
\end{align*}
Hence taking expectations
\begin{align}\label{eq:po2}
    E_{0}^{Q}[g(F_{T})] = g(F_{0}) + \frac{1}{2} E_{0}^{Q} \left[ \int_{0}^{T} g''(F_{t}) F_{t}^{2} \sigma_{t}^{2} \diff t \right]
\end{align}

\textit{Note: In the paper the second expectation is $E_{t}^{Q}$ is that a typo?}\\

setting \ref{eq:po1} and \ref{eq:po2} equal we obtain
\begin{align}\label{eq:narela}
    E_{0}^{Q} \left[ \int_{0}^{T} g''(F_{t}) F_{t}^{2} \sigma_{t}^{2} \diff t \right] = 2 \int_{0}^{\infty} g''(K)M_{0}(K) \diff K
\end{align}

\subsubsection{Introduction of Barrier contracts}

\textbf{Contract for deriving MFIV}\\
\textit{$\rightarrow$ show how we can derive the MFIV from a barrier volatility contract}\\

Consider a contract that pays at time T the realized variance only when the futures price lies below the barrier
\begin{align*}
BIVAR_{B}(0,T) = \int_{0}^{T} \sigma_{t}^{2} I_{t}(B) \diff t
\end{align*}
with 
\begin{align*}
I_{t} = I_{t}(B) = 1[F_{t} \leq B]
\end{align*}
If $B \rightarrow \infty$ the payoff approaches the standard integrated variance
\begin{align*}
IV AR(0,T) = \int_{0}^{T} \sigma_{t}^{2} \diff t
\end{align*}

Suppose that the function $g(F_{T})$ is choosen as
\begin{align*}
g(F_{T}) = g(F_{T};B) = \left (- ln\frac{F_{T}}{B} + \frac{F_{T}}{B} - 1 \right) I_{T}
\end{align*}
The NoA value of the contract can be derived from equation \ref{eq:narela} as
\begin{align*}
BV AR_{0}(B) = E_{0}^{Q} \left [\int_{0}^{T} \sigma_{t}^{2} I_{t} dt \right] = 2 \int_{0}^{B} \frac{M_{0}(K)}{K^{2}} \diff K
\end{align*}

The square-root can be interpreted as the option-implied barrier volatility
\begin{align*}
BIV_{0}(B) = \sqrt{\int_{0}^{B} \frac{M_{0}(K)}{K^{2}} \diff K}
\end{align*}

\textit{Note: this is the formula in our paper to approximate $E_{t}^{Q}[RV]$, eq (6),(7). Hence we are missing the ``proof'' of the step that RV can be estimated with the integrated variance, for this check for example: Andersen, Bollerslev (2002): Parametric and Nonparametric volatility measurement.}\\

In the limiting case of $B = \infty$ the barrier implied volatiltiy coincides with the $MFIV_{0}$ (developed by Dupire (1993), Neuberger (1994)).\\


\textbf{Contract for deriving CIV}\\
\textit{$\rightarrow$ show how we can derive the CV from a corridor volatility contract}\\

The contract which pays corridor variance can be constructed from two barrier variance contracts with different barriers, here $B_{1}$ and $B_{2}$ lower and upper barrier
\begin{align*}
CIV AR_{B_{1}, B_{2}}(0,T) = \int_{0}^{T} \sigma_{t}^{2}I_{t}(B_{1}, B_{2}) \diff t
\end{align*}
with 
\begin{align*}
I_{t}(B_{1}, B_{2}) = I_{t} = 1[B_{1} \leq F_{t} \leq B_{2}]
\end{align*}

Hence the ocntracts pays the \textit{corridor variance} when the futures price lies between the barriers. The value of the contract is
\begin{align*}
CV AR_{0}(B_{1}, B_{2}) = E_{0}^{Q} \left[ \int_{0}^{\infty} \sigma_{t}^{2} I_{t} \diff t \right] = 2 \int_{B_{1}}^{B_{2}} \frac{M_{0}(K)}{K^{2}} \diff K
\end{align*}

In the limiting case where $\Delta (B_{1}, B_{2}) \rightarrow 0$ the value of the contract approches the value of a contract that pays the future variance only \textit{along} a strike. 
\begin{align*}
SV AR_{0}(B) = \lim_{\Delta B \rightarrow 0} \frac{B}{\Delta B} CV AR_{0}(B, B + \Delta B) = 2 \frac{M_{0}(B)}{B}
\end{align*}

\end{document}