%!TEX root = ../Main.tex

\section{Discussions}\label{sec:chapter5}

Overall, the paper presents a very comprehensive and innovative investigation of the predictability of excess returns. The authors fill the gap to investigate, why the well-known relationship between the variance risk premium and excess return that holds in the long-run can sometimes not be observed in the short run. The authors also combine decomposing returns and using option-implied measures, and they implement a comprehensive list of econometric models and construct an economic equilibrium to shed light on the aforementioned question.\\

Having mentioned the above, we however also found some aspects that complicated the replication for us and that we would hence like to point out. Our main challenge was that part of their analysis was not described in great detail. This concerned not only data treatment but also the implementation of the models. As it is the case with using option data for risk-neutral expectation, a main restriction of the analysis lies in the data treatment, hence this was a significant challenge to our replication. Moreover, we were surprised that they calculated their models using overlapping data, given the fact that their data availability stretched a period from 1996 to 2015.\\

To give an outlook on further models that could be implemented, it might be promising to implement their models using non-overlapping data, hence reducing the problem of large correlation in the variables though it would possibly require limiting the analysis to an aggregation and forecasting horizon of a quarter at the most. Moreover, an investigation of different thresholds $\kappa$ could yield informative results.\\

As a larger extension of their approach, it would be interesting to use machine-learning techniques to the question investigated. In the robustness section of the paper, the authors add other well-known predictors of the equity return to their model. In these models particularly it would be interesting to use model-selection techniques that put a penalty on new parameters added, such as lasso or ridge regression. Moreover, we thought about whether it would be possible to use unsupervised learning techniques to separate the variance risk premium, not assuming that upside and downside is the most informative separation for predicting equity returns.
