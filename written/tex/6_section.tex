%!TEX root = ../Main.tex

\section{Conclusion}\label{sec:chapter6}

We analyzed the paper 'Downside Variance Risk Premia' by \textit{B. Fenou, M.R. Jahan-Parvar and C. Okou [2015]}, we discussed their methodology and investigated whether their results hold in a different time period (2007-2017). We computed the physical expectation of the realized variance using historical intraday returns and the risk-free one working on European call and put option data observed in the market. We analyzed the predictability of excess returns using the variance risk premium, realized variance, and implied volatility. Overall, downside variance risk premium has a higher explanatory power for future excess returns and risk-neutral expectations contribute stronger to the predictability than realized measures.
Finally, we proceeded to evaluate the one-step ahead prediction accuracy of our models through expanding window regressions. We have found low RMSE values across all construction horizons with peak performance for $h=2$. The best results where obtained when using the downside variance risk premia as predictor. 